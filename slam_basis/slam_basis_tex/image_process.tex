\chapter{数字图像处理}

\section{颜色空间转换}
1. RGB to Gray (CV\_BGR2GRAY) :
\begin{equation} Y = 0.299*R + 0.587*G + 0.114*B \end{equation}

\section{图像金字塔}
\begin{itemize}
\item 均值金字塔:2*2邻域均值滤波
\item 高斯金字塔:向下降采样图像(4层),高斯核5*5
\item 拉普拉斯金字塔
\end{itemize}

\section{FAST Algorithm for Corner Detection}

\href{https://docs.opencv.org/3.0-beta/doc/py_tutorials/py_feature2d/py_fast/py_fast.html}{FAST Algorithm for Corner Detection}

\subsection{Features from Accelerated Segment Test}
检测**局部像素灰度**变化明显的地方。
\begin{enumerate}   
\item 在图像中选取像素p,假设它的亮度为Ip;
\item 设置一个阈值T;
\item 以像素p为中心,选取半径为3的圆上的16个像素;
\item 假设选取的圆上有连续的N个点的亮度大于Ip+T或Ip-T,则该点p可被认为是特征点(N通常取12,即为FAST-12,其他常用的N取值有9和11,分别被成为FAST-9和FAST-11);
\item 循环以上四步;
\end{enumerate}

%\write18{wget https://docs.opencv.org/3.0-beta/_images/fast_speedtest.jpg}
%\includegraphics{fast_speedtest.jpg}

\subsection{Non-maximal Suppression}
FAST角点经常出现“扎堆”的情况,通过非极大值抑制,在一定区域内仅保留响应极大值的角点,避免角点集中的问题。


\section{Shi-Tomas}
\begin{itemize}
\item \href{https://docs.opencv.org/2.4/doc/tutorials/features2d/trackingmotion/harris_detector/harris_detector.html}{Harris corner detector}
\item \href{https://docs.opencv.org/3.0-beta/doc/py_tutorials/py_feature2d/py_shi_tomasi/py_shi_tomasi.html}{Shi-Tomasi Corner Detector \& Good Features to Track}
\end{itemize}
计算方法:\newline
\begin{enumerate}  
\item 根据Harris角点检测方法,构建梯度协方差矩阵M
\begin{equation}
M =  
\left[\begin{array}{cc}
{I_x}^2&I_xI_y\\I_xI_y&{I_y}^2
\end{array}\right]  
\end{equation}
\item 求矩阵M的特征值
\begin{equation} det(\lambda E - M) = 0 \end{equation}
\item Shi-Tomas分数即为最小特征值
\begin{equation} R=min(\lambda_1,\lambda_2) \end{equation}
\end{enumerate}


\section{SSD}
Sum of Squared Distance: \newline
\begin{equation} {D(I_A,I_B)}_{SSD} = \sum_{x,y}[{I_A}_{(x,y)}-{I_B}_{(x,y)}]^2  \end{equation}


\section{块匹配}
\begin{enumerate}  
\item 假设图像I1和图像I2,分别对应的角点为p1i和p2j,在图像I2角点中找到与图像I1对应的角点;
\item 以角点p1i为中心,在图像I1中提取9*9的像素块作为模板图像T1i;
\item 在图像I2中p1i点周围(以角点p1i为中心20*20的像素 范围)查找所有的角点p2jm(m<=n,n为该范围内角点数);
\item 遍历所有的角点p2jm,以角点p2jm为中心,在图像I2中提取9*9的像素块,计算该像素块与T1i的SSD;
\item SSD最小对应的角点p2jm,即为图像I2中与图像I1中角点p1i对应的匹配角点;
\item 循环执行以上5步,查找图像I2中与图像I1对应的所有匹配角点;
\end{enumerate}