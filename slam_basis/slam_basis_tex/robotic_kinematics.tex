\chapter{机器人运动学}

\section{李群和李代数}~\citep{Gao2017SLAM}

\subsection{特殊正交群}
\begin{equation}
SO(n) = \{ R \in \mathbb{R}^{n \times n} | RR^\top = I, det(R) = 1\}
\end{equation}

\begin{equation} R = exp({\xi}^{\wedge}) \end{equation}
\begin{equation} \xi = ln(R)^{\vee} \end{equation}

注:${\xi}^{\wedge}$为$\xi$的反对称矩阵

\subsection{特殊欧式群}
\begin{equation}
SE(3) = \{ T = \begin{bmatrix} R & t \\ 0^\top & 1 \end{bmatrix} \in \mathbb{R}^{4 \times 4} | R \in SO(3), t \in \mathbb{R}^{3}\}
\end{equation}


\section{欧式变换}
Translate by -C(align origins), Rotate to align axes:
\begin{equation} P_c = R (P_w - C) = RP_w - RC = RP_w + t \end{equation}

\subsection{旋转的表示}

\subsubsection{旋转矩阵} 
\begin{equation}
R =  \begin{bmatrix} r_{11} &  r_{12} & r_{13} \\  r_{21} & r_{22} & r_{23} \\ r_{31} & r_{32} & r_{33} \end{bmatrix} 
\end{equation}

\subsubsection{旋转向量} 
\begin{equation} \xi = ln(R)^{\vee} \end{equation}
\begin{enumerate}
\item 旋转轴:\begin{equation} \frac{\xi}{||\xi||}  \end{equation}
\item 旋转角:\begin{equation} ||\xi||  \end{equation}
\end{enumerate}

\subsubsection{四元数} 
2D旋转:单位复数可用来表示2D旋转。  \\
\begin{equation} 
z = a + b\vec{i} = r ( cos\theta + sin\theta\vec{i} ) = e^{\theta \vec{i}}, r = |z| = 1
\end{equation}
3D旋转:单位四元数才可表示3D旋转,四元数是复数的扩充,在表示旋转前需要进行归一化。 \\
\begin{equation} 
Quarternion = q_0 + q_1\vec{i} + q_2\vec{j} + q_3\vec{k}  
\end{equation} 
四元数可以在保证效率的同时,减小矩阵1/4的内存占有量,同时又能避免欧拉角的万向锁问题。

\subsubsection{欧拉角} 


\subsection{平移}

\subsection{缩放}