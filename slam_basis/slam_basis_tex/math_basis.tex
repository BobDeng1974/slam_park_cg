\chapter{数学基础}

\section{线性代数与矩阵分析}

\subsection{Special Matrices}
\begin{enumerate}
\item Jacobian Matrix
\item Hessian Matrix
\item Covariance Matrix
\end{enumerate}

\subsection{Matrix Decomposition}

\subsubsection{EVD} 

\subsubsection{SVD} 

\subsubsection{QR} 
a decomposition of a matrix A into a product A = QR of an orthogonal matrix Q and an upper triangular matrix R.  \\

QR decomposition is often used to solve the linear least squares problem, and is the basis for a particular eigenvalue algorithm. \\

Any real square matrix A may be decomposed as A = QR, where Q is an orthogonal matrix (its columns are orthogonal unit vectors meaning QTQ = I) and R is an upper triangular matrix (also called right triangular matrix).  If A is invertible, then the factorization is unique if we require the diagonal elements of R to be positive. \\

If instead A is a complex square matrix, then there is a decomposition A = QR where Q is a unitary matrix (so Q*Q = I).

\subsubsection{LU} 

\subsubsection{Cholesky} 
a decomposition of a Hermitian, positive-definite matrix into the product of a lower triangular matrix and its conjugate transpose, which is useful for efficient numerical solutions, e.g. Monte Carlo simulations. It was discovered by André-Louis Cholesky for real matrices. When it is applicable, the Cholesky decomposition is roughly twice as efficient as the LU decomposition for solving systems of linear equations.  \\

The Cholesky decomposition of a Hermitian positive-definite matrix A is a decomposition of the form
\begin{equation}
A = LL^{*}
\end{equation}
where L is a lower triangular matrix with real and positive diagonal entries, and L* denotes the conjugate transpose of L. Every Hermitian positive-definite matrix (and thus also every real-valued symmetric positive-definite matrix) has a unique Cholesky decomposition. \\

If the matrix A is Hermitian and positive semi-definite, then it still has a decomposition of the form A = LL* if the diagonal entries of L are allowed to be zero. \\

When A has real entries, L has real entries as well, and the factorization may be written A = LLT. \\

The Cholesky decomposition is unique when A is positive definite; there is only one lower triangular matrix L with strictly positive diagonal entries such that A = LL*. However, the decomposition need not be unique when A is positive semidefinite. \\

The converse holds trivially: if A can be written as LL* for some invertible L, lower triangular or otherwise, then A is Hermitian and positive definite.

\subsubsection{Summary}
\begin{enumerate}
\item 在实际应用中,因为数值稳定性的要求 ,dense matrix 往往用QR求解 ,对于 大型的稀疏矩阵则多用Cholesky分解
\end{enumerate}


\section{数值分析}

线性插值  

双线性插值  

抛物线拟合  


\section{概率论与数理统计}

Probability Theory and Statistics

\subsection{MLE(Maximum Likelihood Estimate)}
\subsection{OLS(Ordinary Least Squares)}
\subsection{RANSAC(RANdom SAmple Consensus)}
\subsection{M-Estimator}


\section{优化理论}

\subsection{Gauss-Newton}
\subsection{Levenberg-Marquardt}
\subsection{ESM}
\subsection{Bundle Adjustment}
\subsection{图优化}
g2o