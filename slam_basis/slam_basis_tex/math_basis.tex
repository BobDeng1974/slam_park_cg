\chapter{数学基础}

\section{Matrix}

\subsection{Special Matrices}
\begin{enumerate}
\item Jacobian Matrix
\item Hessian Matrix
\item Covariance Matrix
\end{enumerate}

\subsection{Matrix Decomposition}

\subsubsection{EVD} 

\subsubsection{SVD} 

\subsubsection{QR} 
a decomposition of a matrix A into a product A = QR of an orthogonal matrix Q and an upper triangular matrix R.  \\

QR decomposition is often used to solve the linear least squares problem, and is the basis for a particular eigenvalue algorithm. \\

Any real square matrix A may be decomposed as A = QR, where Q is an orthogonal matrix (its columns are orthogonal unit vectors meaning QTQ = I) and R is an upper triangular matrix (also called right triangular matrix).  If A is invertible, then the factorization is unique if we require the diagonal elements of R to be positive. \\

If instead A is a complex square matrix, then there is a decomposition A = QR where Q is a unitary matrix (so Q*Q = I).

\subsubsection{LU} 

\subsubsection{Cholesky} 
a decomposition of a Hermitian, positive-definite matrix into the product of a lower triangular matrix and its conjugate transpose, which is useful for efficient numerical solutions, e.g. Monte Carlo simulations. It was discovered by André-Louis Cholesky for real matrices. When it is applicable, the Cholesky decomposition is roughly twice as efficient as the LU decomposition for solving systems of linear equations.  \\

The Cholesky decomposition of a Hermitian positive-definite matrix A is a decomposition of the form
\begin{equation}
A = LL^{*}
\end{equation}
where L is a lower triangular matrix with real and positive diagonal entries, and L* denotes the conjugate transpose of L. Every Hermitian positive-definite matrix (and thus also every real-valued symmetric positive-definite matrix) has a unique Cholesky decomposition. \\

If the matrix A is Hermitian and positive semi-definite, then it still has a decomposition of the form A = LL* if the diagonal entries of L are allowed to be zero. \\

When A has real entries, L has real entries as well, and the factorization may be written A = LLT. \\

The Cholesky decomposition is unique when A is positive definite; there is only one lower triangular matrix L with strictly positive diagonal entries such that A = LL*. However, the decomposition need not be unique when A is positive semidefinite. \\

The converse holds trivially: if A can be written as LL* for some invertible L, lower triangular or otherwise, then A is Hermitian and positive definite.

\subsubsection{Summary}
\begin{enumerate}
\item 在实际应用中,因为数值稳定性的要求 ,dense matrix 往往用QR求解 ,对于 大型的稀疏矩阵则多用Cholesky分解
\end{enumerate}


\section{李群和李代数}

\subsection{特殊正交群}
\begin{equation}
SO(n) = \{ R \in \mathbb{R}^{n \times n} | RR^\top = I, det(R) = 1\}
\end{equation}

\begin{equation} R = exp({\xi}^{\wedge}) \end{equation}
\begin{equation} \xi = ln(R)^{\vee} \end{equation}

注:${\xi}^{\wedge}$为$\xi$的反对称矩阵

\subsection{特殊欧式群}
\begin{equation}
SE(3) = \{ T = \begin{bmatrix} R & t \\ 0^\top & 1 \end{bmatrix} \in \mathbb{R}^{4 \times 4} | R \in SO(3), t \in \mathbb{R}^{3}\}
\end{equation}


\section{欧式变换}
Translate by -C(align origins), Rotate to align axes:
\begin{equation} P_c = R (P_w - C) = RP_w - RC = RP_w + t \end{equation}

\subsection{旋转的表示}

\subsubsection{旋转矩阵} 
\begin{equation}
R =  \begin{bmatrix} r_{11} &  r_{12} & r_{13} \\  r_{21} & r_{22} & r_{23} \\ r_{31} & r_{32} & r_{33} \end{bmatrix} 
\end{equation}

\subsubsection{旋转向量} 
\begin{equation} \xi = ln(R)^{\vee} \end{equation}
\begin{enumerate}
\item 旋转轴:\begin{equation} \frac{\xi}{||\xi||}  \end{equation}
\item 旋转角:\begin{equation} ||\xi||  \end{equation}
\end{enumerate}

\subsubsection{四元数} 
2D旋转:单位复数可用来表示2D旋转。  \\
\begin{equation} 
z = a + b\vec{i} = r ( cos\theta + sin\theta\vec{i} ) = e^{\theta \vec{i}}, r = |z| = 1
\end{equation}
3D旋转:单位四元数才可表示3D旋转,四元数是复数的扩充,在表示旋转前需要进行归一化。 \\
\begin{equation} 
Quarternion = q_0 + q_1\vec{i} + q_2\vec{j} + q_3\vec{k}  
\end{equation} 
四元数可以在保证效率的同时,减小矩阵1/4的内存占有量,同时又能避免欧拉角的万向锁问题。

\subsubsection{欧拉角} 


\subsection{平移}

\subsection{缩放}



\section{对极几何}

\subsection{Essential Matrix}
\begin{equation} E = t \wedge R \end{equation}
\begin{equation} {x_2}^\top \cdot E \cdot x_1 = 0 \end{equation}
 
\subsection{Foundamental Matrix}
\begin{equation} F = K^{-\top} E K^{-1} \end{equation}

\subsection{Homography Matrix}
\begin{equation} x_2 = H \cdot x_1 \end{equation}


\section{Probability Theory and Statistics}
\subsection{MLE(Maximum Likelihood Estimate)}
\subsection{OLS(Ordinary Least Squares)}
\subsection{RANSAC(RANdom SAmple Consensus)}
\subsection{M-Estimator}

\section{优化理论}
\subsection{Gauss-Newton}
\subsection{Levenberg-Marquardt}
\subsection{ESM}
\subsection{Bundle Adjustment}
\subsection{图优化}
g2o